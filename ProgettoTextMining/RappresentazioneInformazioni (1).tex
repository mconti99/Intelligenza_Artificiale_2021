\documentclass[11pt]{article}

    \usepackage[breakable]{tcolorbox}
    \usepackage{parskip} % Stop auto-indenting (to mimic markdown behaviour)
    
    \usepackage{iftex}
    \ifPDFTeX
    	\usepackage[T1]{fontenc}
    	\usepackage{mathpazo}
    \else
    	\usepackage{fontspec}
    \fi

    % Basic figure setup, for now with no caption control since it's done
    % automatically by Pandoc (which extracts ![](path) syntax from Markdown).
    \usepackage{graphicx}
    % Maintain compatibility with old templates. Remove in nbconvert 6.0
    \let\Oldincludegraphics\includegraphics
    % Ensure that by default, figures have no caption (until we provide a
    % proper Figure object with a Caption API and a way to capture that
    % in the conversion process - todo).
    \usepackage{caption}
    \DeclareCaptionFormat{nocaption}{}
    \captionsetup{format=nocaption,aboveskip=0pt,belowskip=0pt}

    \usepackage{float}
    \floatplacement{figure}{H} % forces figures to be placed at the correct location
    \usepackage{xcolor} % Allow colors to be defined
    \usepackage{enumerate} % Needed for markdown enumerations to work
    \usepackage{geometry} % Used to adjust the document margins
    \usepackage{amsmath} % Equations
    \usepackage{amssymb} % Equations
    \usepackage{textcomp} % defines textquotesingle
    % Hack from http://tex.stackexchange.com/a/47451/13684:
    \AtBeginDocument{%
        \def\PYZsq{\textquotesingle}% Upright quotes in Pygmentized code
    }
    \usepackage{upquote} % Upright quotes for verbatim code
    \usepackage{eurosym} % defines \euro
    \usepackage[mathletters]{ucs} % Extended unicode (utf-8) support
    \usepackage{fancyvrb} % verbatim replacement that allows latex
    \usepackage{grffile} % extends the file name processing of package graphics 
                         % to support a larger range
    \makeatletter % fix for old versions of grffile with XeLaTeX
    \@ifpackagelater{grffile}{2019/11/01}
    {
      % Do nothing on new versions
    }
    {
      \def\Gread@@xetex#1{%
        \IfFileExists{"\Gin@base".bb}%
        {\Gread@eps{\Gin@base.bb}}%
        {\Gread@@xetex@aux#1}%
      }
    }
    \makeatother
    \usepackage[Export]{adjustbox} % Used to constrain images to a maximum size
    \adjustboxset{max size={0.9\linewidth}{0.9\paperheight}}

    % The hyperref package gives us a pdf with properly built
    % internal navigation ('pdf bookmarks' for the table of contents,
    % internal cross-reference links, web links for URLs, etc.)
    \usepackage{hyperref}
    % The default LaTeX title has an obnoxious amount of whitespace. By default,
    % titling removes some of it. It also provides customization options.
    \usepackage{titling}
    \usepackage{longtable} % longtable support required by pandoc >1.10
    \usepackage{booktabs}  % table support for pandoc > 1.12.2
    \usepackage[inline]{enumitem} % IRkernel/repr support (it uses the enumerate* environment)
    \usepackage[normalem]{ulem} % ulem is needed to support strikethroughs (\sout)
                                % normalem makes italics be italics, not underlines
    \usepackage{mathrsfs}
    

    
    % Colors for the hyperref package
    \definecolor{urlcolor}{rgb}{0,.145,.698}
    \definecolor{linkcolor}{rgb}{.71,0.21,0.01}
    \definecolor{citecolor}{rgb}{.12,.54,.11}

    % ANSI colors
    \definecolor{ansi-black}{HTML}{3E424D}
    \definecolor{ansi-black-intense}{HTML}{282C36}
    \definecolor{ansi-red}{HTML}{E75C58}
    \definecolor{ansi-red-intense}{HTML}{B22B31}
    \definecolor{ansi-green}{HTML}{00A250}
    \definecolor{ansi-green-intense}{HTML}{007427}
    \definecolor{ansi-yellow}{HTML}{DDB62B}
    \definecolor{ansi-yellow-intense}{HTML}{B27D12}
    \definecolor{ansi-blue}{HTML}{208FFB}
    \definecolor{ansi-blue-intense}{HTML}{0065CA}
    \definecolor{ansi-magenta}{HTML}{D160C4}
    \definecolor{ansi-magenta-intense}{HTML}{A03196}
    \definecolor{ansi-cyan}{HTML}{60C6C8}
    \definecolor{ansi-cyan-intense}{HTML}{258F8F}
    \definecolor{ansi-white}{HTML}{C5C1B4}
    \definecolor{ansi-white-intense}{HTML}{A1A6B2}
    \definecolor{ansi-default-inverse-fg}{HTML}{FFFFFF}
    \definecolor{ansi-default-inverse-bg}{HTML}{000000}

    % common color for the border for error outputs.
    \definecolor{outerrorbackground}{HTML}{FFDFDF}

    % commands and environments needed by pandoc snippets
    % extracted from the output of `pandoc -s`
    \providecommand{\tightlist}{%
      \setlength{\itemsep}{0pt}\setlength{\parskip}{0pt}}
    \DefineVerbatimEnvironment{Highlighting}{Verbatim}{commandchars=\\\{\}}
    % Add ',fontsize=\small' for more characters per line
    \newenvironment{Shaded}{}{}
    \newcommand{\KeywordTok}[1]{\textcolor[rgb]{0.00,0.44,0.13}{\textbf{{#1}}}}
    \newcommand{\DataTypeTok}[1]{\textcolor[rgb]{0.56,0.13,0.00}{{#1}}}
    \newcommand{\DecValTok}[1]{\textcolor[rgb]{0.25,0.63,0.44}{{#1}}}
    \newcommand{\BaseNTok}[1]{\textcolor[rgb]{0.25,0.63,0.44}{{#1}}}
    \newcommand{\FloatTok}[1]{\textcolor[rgb]{0.25,0.63,0.44}{{#1}}}
    \newcommand{\CharTok}[1]{\textcolor[rgb]{0.25,0.44,0.63}{{#1}}}
    \newcommand{\StringTok}[1]{\textcolor[rgb]{0.25,0.44,0.63}{{#1}}}
    \newcommand{\CommentTok}[1]{\textcolor[rgb]{0.38,0.63,0.69}{\textit{{#1}}}}
    \newcommand{\OtherTok}[1]{\textcolor[rgb]{0.00,0.44,0.13}{{#1}}}
    \newcommand{\AlertTok}[1]{\textcolor[rgb]{1.00,0.00,0.00}{\textbf{{#1}}}}
    \newcommand{\FunctionTok}[1]{\textcolor[rgb]{0.02,0.16,0.49}{{#1}}}
    \newcommand{\RegionMarkerTok}[1]{{#1}}
    \newcommand{\ErrorTok}[1]{\textcolor[rgb]{1.00,0.00,0.00}{\textbf{{#1}}}}
    \newcommand{\NormalTok}[1]{{#1}}
    
    % Additional commands for more recent versions of Pandoc
    \newcommand{\ConstantTok}[1]{\textcolor[rgb]{0.53,0.00,0.00}{{#1}}}
    \newcommand{\SpecialCharTok}[1]{\textcolor[rgb]{0.25,0.44,0.63}{{#1}}}
    \newcommand{\VerbatimStringTok}[1]{\textcolor[rgb]{0.25,0.44,0.63}{{#1}}}
    \newcommand{\SpecialStringTok}[1]{\textcolor[rgb]{0.73,0.40,0.53}{{#1}}}
    \newcommand{\ImportTok}[1]{{#1}}
    \newcommand{\DocumentationTok}[1]{\textcolor[rgb]{0.73,0.13,0.13}{\textit{{#1}}}}
    \newcommand{\AnnotationTok}[1]{\textcolor[rgb]{0.38,0.63,0.69}{\textbf{\textit{{#1}}}}}
    \newcommand{\CommentVarTok}[1]{\textcolor[rgb]{0.38,0.63,0.69}{\textbf{\textit{{#1}}}}}
    \newcommand{\VariableTok}[1]{\textcolor[rgb]{0.10,0.09,0.49}{{#1}}}
    \newcommand{\ControlFlowTok}[1]{\textcolor[rgb]{0.00,0.44,0.13}{\textbf{{#1}}}}
    \newcommand{\OperatorTok}[1]{\textcolor[rgb]{0.40,0.40,0.40}{{#1}}}
    \newcommand{\BuiltInTok}[1]{{#1}}
    \newcommand{\ExtensionTok}[1]{{#1}}
    \newcommand{\PreprocessorTok}[1]{\textcolor[rgb]{0.74,0.48,0.00}{{#1}}}
    \newcommand{\AttributeTok}[1]{\textcolor[rgb]{0.49,0.56,0.16}{{#1}}}
    \newcommand{\InformationTok}[1]{\textcolor[rgb]{0.38,0.63,0.69}{\textbf{\textit{{#1}}}}}
    \newcommand{\WarningTok}[1]{\textcolor[rgb]{0.38,0.63,0.69}{\textbf{\textit{{#1}}}}}
    
    
    % Define a nice break command that doesn't care if a line doesn't already
    % exist.
    \def\br{\hspace*{\fill} \\* }
    % Math Jax compatibility definitions
    \def\gt{>}
    \def\lt{<}
    \let\Oldtex\TeX
    \let\Oldlatex\LaTeX
    \renewcommand{\TeX}{\textrm{\Oldtex}}
    \renewcommand{\LaTeX}{\textrm{\Oldlatex}}
    % Document parameters
    % Document title
    \title{RappresentazioneInformazioni (1)}
    
    
    
    
    
% Pygments definitions
\makeatletter
\def\PY@reset{\let\PY@it=\relax \let\PY@bf=\relax%
    \let\PY@ul=\relax \let\PY@tc=\relax%
    \let\PY@bc=\relax \let\PY@ff=\relax}
\def\PY@tok#1{\csname PY@tok@#1\endcsname}
\def\PY@toks#1+{\ifx\relax#1\empty\else%
    \PY@tok{#1}\expandafter\PY@toks\fi}
\def\PY@do#1{\PY@bc{\PY@tc{\PY@ul{%
    \PY@it{\PY@bf{\PY@ff{#1}}}}}}}
\def\PY#1#2{\PY@reset\PY@toks#1+\relax+\PY@do{#2}}

\@namedef{PY@tok@w}{\def\PY@tc##1{\textcolor[rgb]{0.73,0.73,0.73}{##1}}}
\@namedef{PY@tok@c}{\let\PY@it=\textit\def\PY@tc##1{\textcolor[rgb]{0.25,0.50,0.50}{##1}}}
\@namedef{PY@tok@cp}{\def\PY@tc##1{\textcolor[rgb]{0.74,0.48,0.00}{##1}}}
\@namedef{PY@tok@k}{\let\PY@bf=\textbf\def\PY@tc##1{\textcolor[rgb]{0.00,0.50,0.00}{##1}}}
\@namedef{PY@tok@kp}{\def\PY@tc##1{\textcolor[rgb]{0.00,0.50,0.00}{##1}}}
\@namedef{PY@tok@kt}{\def\PY@tc##1{\textcolor[rgb]{0.69,0.00,0.25}{##1}}}
\@namedef{PY@tok@o}{\def\PY@tc##1{\textcolor[rgb]{0.40,0.40,0.40}{##1}}}
\@namedef{PY@tok@ow}{\let\PY@bf=\textbf\def\PY@tc##1{\textcolor[rgb]{0.67,0.13,1.00}{##1}}}
\@namedef{PY@tok@nb}{\def\PY@tc##1{\textcolor[rgb]{0.00,0.50,0.00}{##1}}}
\@namedef{PY@tok@nf}{\def\PY@tc##1{\textcolor[rgb]{0.00,0.00,1.00}{##1}}}
\@namedef{PY@tok@nc}{\let\PY@bf=\textbf\def\PY@tc##1{\textcolor[rgb]{0.00,0.00,1.00}{##1}}}
\@namedef{PY@tok@nn}{\let\PY@bf=\textbf\def\PY@tc##1{\textcolor[rgb]{0.00,0.00,1.00}{##1}}}
\@namedef{PY@tok@ne}{\let\PY@bf=\textbf\def\PY@tc##1{\textcolor[rgb]{0.82,0.25,0.23}{##1}}}
\@namedef{PY@tok@nv}{\def\PY@tc##1{\textcolor[rgb]{0.10,0.09,0.49}{##1}}}
\@namedef{PY@tok@no}{\def\PY@tc##1{\textcolor[rgb]{0.53,0.00,0.00}{##1}}}
\@namedef{PY@tok@nl}{\def\PY@tc##1{\textcolor[rgb]{0.63,0.63,0.00}{##1}}}
\@namedef{PY@tok@ni}{\let\PY@bf=\textbf\def\PY@tc##1{\textcolor[rgb]{0.60,0.60,0.60}{##1}}}
\@namedef{PY@tok@na}{\def\PY@tc##1{\textcolor[rgb]{0.49,0.56,0.16}{##1}}}
\@namedef{PY@tok@nt}{\let\PY@bf=\textbf\def\PY@tc##1{\textcolor[rgb]{0.00,0.50,0.00}{##1}}}
\@namedef{PY@tok@nd}{\def\PY@tc##1{\textcolor[rgb]{0.67,0.13,1.00}{##1}}}
\@namedef{PY@tok@s}{\def\PY@tc##1{\textcolor[rgb]{0.73,0.13,0.13}{##1}}}
\@namedef{PY@tok@sd}{\let\PY@it=\textit\def\PY@tc##1{\textcolor[rgb]{0.73,0.13,0.13}{##1}}}
\@namedef{PY@tok@si}{\let\PY@bf=\textbf\def\PY@tc##1{\textcolor[rgb]{0.73,0.40,0.53}{##1}}}
\@namedef{PY@tok@se}{\let\PY@bf=\textbf\def\PY@tc##1{\textcolor[rgb]{0.73,0.40,0.13}{##1}}}
\@namedef{PY@tok@sr}{\def\PY@tc##1{\textcolor[rgb]{0.73,0.40,0.53}{##1}}}
\@namedef{PY@tok@ss}{\def\PY@tc##1{\textcolor[rgb]{0.10,0.09,0.49}{##1}}}
\@namedef{PY@tok@sx}{\def\PY@tc##1{\textcolor[rgb]{0.00,0.50,0.00}{##1}}}
\@namedef{PY@tok@m}{\def\PY@tc##1{\textcolor[rgb]{0.40,0.40,0.40}{##1}}}
\@namedef{PY@tok@gh}{\let\PY@bf=\textbf\def\PY@tc##1{\textcolor[rgb]{0.00,0.00,0.50}{##1}}}
\@namedef{PY@tok@gu}{\let\PY@bf=\textbf\def\PY@tc##1{\textcolor[rgb]{0.50,0.00,0.50}{##1}}}
\@namedef{PY@tok@gd}{\def\PY@tc##1{\textcolor[rgb]{0.63,0.00,0.00}{##1}}}
\@namedef{PY@tok@gi}{\def\PY@tc##1{\textcolor[rgb]{0.00,0.63,0.00}{##1}}}
\@namedef{PY@tok@gr}{\def\PY@tc##1{\textcolor[rgb]{1.00,0.00,0.00}{##1}}}
\@namedef{PY@tok@ge}{\let\PY@it=\textit}
\@namedef{PY@tok@gs}{\let\PY@bf=\textbf}
\@namedef{PY@tok@gp}{\let\PY@bf=\textbf\def\PY@tc##1{\textcolor[rgb]{0.00,0.00,0.50}{##1}}}
\@namedef{PY@tok@go}{\def\PY@tc##1{\textcolor[rgb]{0.53,0.53,0.53}{##1}}}
\@namedef{PY@tok@gt}{\def\PY@tc##1{\textcolor[rgb]{0.00,0.27,0.87}{##1}}}
\@namedef{PY@tok@err}{\def\PY@bc##1{{\setlength{\fboxsep}{-\fboxrule}\fcolorbox[rgb]{1.00,0.00,0.00}{1,1,1}{\strut ##1}}}}
\@namedef{PY@tok@kc}{\let\PY@bf=\textbf\def\PY@tc##1{\textcolor[rgb]{0.00,0.50,0.00}{##1}}}
\@namedef{PY@tok@kd}{\let\PY@bf=\textbf\def\PY@tc##1{\textcolor[rgb]{0.00,0.50,0.00}{##1}}}
\@namedef{PY@tok@kn}{\let\PY@bf=\textbf\def\PY@tc##1{\textcolor[rgb]{0.00,0.50,0.00}{##1}}}
\@namedef{PY@tok@kr}{\let\PY@bf=\textbf\def\PY@tc##1{\textcolor[rgb]{0.00,0.50,0.00}{##1}}}
\@namedef{PY@tok@bp}{\def\PY@tc##1{\textcolor[rgb]{0.00,0.50,0.00}{##1}}}
\@namedef{PY@tok@fm}{\def\PY@tc##1{\textcolor[rgb]{0.00,0.00,1.00}{##1}}}
\@namedef{PY@tok@vc}{\def\PY@tc##1{\textcolor[rgb]{0.10,0.09,0.49}{##1}}}
\@namedef{PY@tok@vg}{\def\PY@tc##1{\textcolor[rgb]{0.10,0.09,0.49}{##1}}}
\@namedef{PY@tok@vi}{\def\PY@tc##1{\textcolor[rgb]{0.10,0.09,0.49}{##1}}}
\@namedef{PY@tok@vm}{\def\PY@tc##1{\textcolor[rgb]{0.10,0.09,0.49}{##1}}}
\@namedef{PY@tok@sa}{\def\PY@tc##1{\textcolor[rgb]{0.73,0.13,0.13}{##1}}}
\@namedef{PY@tok@sb}{\def\PY@tc##1{\textcolor[rgb]{0.73,0.13,0.13}{##1}}}
\@namedef{PY@tok@sc}{\def\PY@tc##1{\textcolor[rgb]{0.73,0.13,0.13}{##1}}}
\@namedef{PY@tok@dl}{\def\PY@tc##1{\textcolor[rgb]{0.73,0.13,0.13}{##1}}}
\@namedef{PY@tok@s2}{\def\PY@tc##1{\textcolor[rgb]{0.73,0.13,0.13}{##1}}}
\@namedef{PY@tok@sh}{\def\PY@tc##1{\textcolor[rgb]{0.73,0.13,0.13}{##1}}}
\@namedef{PY@tok@s1}{\def\PY@tc##1{\textcolor[rgb]{0.73,0.13,0.13}{##1}}}
\@namedef{PY@tok@mb}{\def\PY@tc##1{\textcolor[rgb]{0.40,0.40,0.40}{##1}}}
\@namedef{PY@tok@mf}{\def\PY@tc##1{\textcolor[rgb]{0.40,0.40,0.40}{##1}}}
\@namedef{PY@tok@mh}{\def\PY@tc##1{\textcolor[rgb]{0.40,0.40,0.40}{##1}}}
\@namedef{PY@tok@mi}{\def\PY@tc##1{\textcolor[rgb]{0.40,0.40,0.40}{##1}}}
\@namedef{PY@tok@il}{\def\PY@tc##1{\textcolor[rgb]{0.40,0.40,0.40}{##1}}}
\@namedef{PY@tok@mo}{\def\PY@tc##1{\textcolor[rgb]{0.40,0.40,0.40}{##1}}}
\@namedef{PY@tok@ch}{\let\PY@it=\textit\def\PY@tc##1{\textcolor[rgb]{0.25,0.50,0.50}{##1}}}
\@namedef{PY@tok@cm}{\let\PY@it=\textit\def\PY@tc##1{\textcolor[rgb]{0.25,0.50,0.50}{##1}}}
\@namedef{PY@tok@cpf}{\let\PY@it=\textit\def\PY@tc##1{\textcolor[rgb]{0.25,0.50,0.50}{##1}}}
\@namedef{PY@tok@c1}{\let\PY@it=\textit\def\PY@tc##1{\textcolor[rgb]{0.25,0.50,0.50}{##1}}}
\@namedef{PY@tok@cs}{\let\PY@it=\textit\def\PY@tc##1{\textcolor[rgb]{0.25,0.50,0.50}{##1}}}

\def\PYZbs{\char`\\}
\def\PYZus{\char`\_}
\def\PYZob{\char`\{}
\def\PYZcb{\char`\}}
\def\PYZca{\char`\^}
\def\PYZam{\char`\&}
\def\PYZlt{\char`\<}
\def\PYZgt{\char`\>}
\def\PYZsh{\char`\#}
\def\PYZpc{\char`\%}
\def\PYZdl{\char`\$}
\def\PYZhy{\char`\-}
\def\PYZsq{\char`\'}
\def\PYZdq{\char`\"}
\def\PYZti{\char`\~}
% for compatibility with earlier versions
\def\PYZat{@}
\def\PYZlb{[}
\def\PYZrb{]}
\makeatother


    % For linebreaks inside Verbatim environment from package fancyvrb. 
    \makeatletter
        \newbox\Wrappedcontinuationbox 
        \newbox\Wrappedvisiblespacebox 
        \newcommand*\Wrappedvisiblespace {\textcolor{red}{\textvisiblespace}} 
        \newcommand*\Wrappedcontinuationsymbol {\textcolor{red}{\llap{\tiny$\m@th\hookrightarrow$}}} 
        \newcommand*\Wrappedcontinuationindent {3ex } 
        \newcommand*\Wrappedafterbreak {\kern\Wrappedcontinuationindent\copy\Wrappedcontinuationbox} 
        % Take advantage of the already applied Pygments mark-up to insert 
        % potential linebreaks for TeX processing. 
        %        {, <, #, %, $, ' and ": go to next line. 
        %        _, }, ^, &, >, - and ~: stay at end of broken line. 
        % Use of \textquotesingle for straight quote. 
        \newcommand*\Wrappedbreaksatspecials {% 
            \def\PYGZus{\discretionary{\char`\_}{\Wrappedafterbreak}{\char`\_}}% 
            \def\PYGZob{\discretionary{}{\Wrappedafterbreak\char`\{}{\char`\{}}% 
            \def\PYGZcb{\discretionary{\char`\}}{\Wrappedafterbreak}{\char`\}}}% 
            \def\PYGZca{\discretionary{\char`\^}{\Wrappedafterbreak}{\char`\^}}% 
            \def\PYGZam{\discretionary{\char`\&}{\Wrappedafterbreak}{\char`\&}}% 
            \def\PYGZlt{\discretionary{}{\Wrappedafterbreak\char`\<}{\char`\<}}% 
            \def\PYGZgt{\discretionary{\char`\>}{\Wrappedafterbreak}{\char`\>}}% 
            \def\PYGZsh{\discretionary{}{\Wrappedafterbreak\char`\#}{\char`\#}}% 
            \def\PYGZpc{\discretionary{}{\Wrappedafterbreak\char`\%}{\char`\%}}% 
            \def\PYGZdl{\discretionary{}{\Wrappedafterbreak\char`\$}{\char`\$}}% 
            \def\PYGZhy{\discretionary{\char`\-}{\Wrappedafterbreak}{\char`\-}}% 
            \def\PYGZsq{\discretionary{}{\Wrappedafterbreak\textquotesingle}{\textquotesingle}}% 
            \def\PYGZdq{\discretionary{}{\Wrappedafterbreak\char`\"}{\char`\"}}% 
            \def\PYGZti{\discretionary{\char`\~}{\Wrappedafterbreak}{\char`\~}}% 
        } 
        % Some characters . , ; ? ! / are not pygmentized. 
        % This macro makes them "active" and they will insert potential linebreaks 
        \newcommand*\Wrappedbreaksatpunct {% 
            \lccode`\~`\.\lowercase{\def~}{\discretionary{\hbox{\char`\.}}{\Wrappedafterbreak}{\hbox{\char`\.}}}% 
            \lccode`\~`\,\lowercase{\def~}{\discretionary{\hbox{\char`\,}}{\Wrappedafterbreak}{\hbox{\char`\,}}}% 
            \lccode`\~`\;\lowercase{\def~}{\discretionary{\hbox{\char`\;}}{\Wrappedafterbreak}{\hbox{\char`\;}}}% 
            \lccode`\~`\:\lowercase{\def~}{\discretionary{\hbox{\char`\:}}{\Wrappedafterbreak}{\hbox{\char`\:}}}% 
            \lccode`\~`\?\lowercase{\def~}{\discretionary{\hbox{\char`\?}}{\Wrappedafterbreak}{\hbox{\char`\?}}}% 
            \lccode`\~`\!\lowercase{\def~}{\discretionary{\hbox{\char`\!}}{\Wrappedafterbreak}{\hbox{\char`\!}}}% 
            \lccode`\~`\/\lowercase{\def~}{\discretionary{\hbox{\char`\/}}{\Wrappedafterbreak}{\hbox{\char`\/}}}% 
            \catcode`\.\active
            \catcode`\,\active 
            \catcode`\;\active
            \catcode`\:\active
            \catcode`\?\active
            \catcode`\!\active
            \catcode`\/\active 
            \lccode`\~`\~ 	
        }
    \makeatother

    \let\OriginalVerbatim=\Verbatim
    \makeatletter
    \renewcommand{\Verbatim}[1][1]{%
        %\parskip\z@skip
        \sbox\Wrappedcontinuationbox {\Wrappedcontinuationsymbol}%
        \sbox\Wrappedvisiblespacebox {\FV@SetupFont\Wrappedvisiblespace}%
        \def\FancyVerbFormatLine ##1{\hsize\linewidth
            \vtop{\raggedright\hyphenpenalty\z@\exhyphenpenalty\z@
                \doublehyphendemerits\z@\finalhyphendemerits\z@
                \strut ##1\strut}%
        }%
        % If the linebreak is at a space, the latter will be displayed as visible
        % space at end of first line, and a continuation symbol starts next line.
        % Stretch/shrink are however usually zero for typewriter font.
        \def\FV@Space {%
            \nobreak\hskip\z@ plus\fontdimen3\font minus\fontdimen4\font
            \discretionary{\copy\Wrappedvisiblespacebox}{\Wrappedafterbreak}
            {\kern\fontdimen2\font}%
        }%
        
        % Allow breaks at special characters using \PYG... macros.
        \Wrappedbreaksatspecials
        % Breaks at punctuation characters . , ; ? ! and / need catcode=\active 	
        \OriginalVerbatim[#1,codes*=\Wrappedbreaksatpunct]%
    }
    \makeatother

    % Exact colors from NB
    \definecolor{incolor}{HTML}{303F9F}
    \definecolor{outcolor}{HTML}{D84315}
    \definecolor{cellborder}{HTML}{CFCFCF}
    \definecolor{cellbackground}{HTML}{F7F7F7}
    
    % prompt
    \makeatletter
    \newcommand{\boxspacing}{\kern\kvtcb@left@rule\kern\kvtcb@boxsep}
    \makeatother
    \newcommand{\prompt}[4]{
        {\ttfamily\llap{{\color{#2}[#3]:\hspace{3pt}#4}}\vspace{-\baselineskip}}
    }
    

    
    % Prevent overflowing lines due to hard-to-break entities
    \sloppy 
    % Setup hyperref package
    \hypersetup{
      breaklinks=true,  % so long urls are correctly broken across lines
      colorlinks=true,
      urlcolor=urlcolor,
      linkcolor=linkcolor,
      citecolor=citecolor,
      }
    % Slightly bigger margins than the latex defaults
    
    \geometry{verbose,tmargin=1in,bmargin=1in,lmargin=1in,rmargin=1in}
    
    

\begin{document}
    
    \maketitle
    
    

    
    \hypertarget{estrazione-di-informazioni-testuali-da-pdf}{%
\section{Estrazione di informazioni testuali da
pdf}\label{estrazione-di-informazioni-testuali-da-pdf}}

    \begin{tcolorbox}[breakable, size=fbox, boxrule=1pt, pad at break*=1mm,colback=cellbackground, colframe=cellborder]
\prompt{In}{incolor}{ }{\boxspacing}
\begin{Verbatim}[commandchars=\\\{\}]
\PY{k+kn}{from} \PY{n+nn}{google}\PY{n+nn}{.}\PY{n+nn}{colab} \PY{k+kn}{import} \PY{n}{drive}
\PY{n}{drive}\PY{o}{.}\PY{n}{mount}\PY{p}{(}\PY{l+s+s1}{\PYZsq{}}\PY{l+s+s1}{/content/drive}\PY{l+s+s1}{\PYZsq{}}\PY{p}{)}
\end{Verbatim}
\end{tcolorbox}

    In questa sezione ci occupiamo di effettuare l'operazione di scanning
dei dati che sono caricati da un pdf caricato su drive. Per caricare i
dati contenuti nei pdf utilizziamo gli appositi metodi forniti dalla
libreria pythin PyPDF2:

    \begin{tcolorbox}[breakable, size=fbox, boxrule=1pt, pad at break*=1mm,colback=cellbackground, colframe=cellborder]
\prompt{In}{incolor}{ }{\boxspacing}
\begin{Verbatim}[commandchars=\\\{\}]
\PY{o}{!}pip install PyPDF2
\PY{k+kn}{import} \PY{n+nn}{PyPDF2}
\PY{k+kn}{from} \PY{n+nn}{PyPDF2} \PY{k+kn}{import} \PY{n}{PdfFileReader}
\end{Verbatim}
\end{tcolorbox}

    Apertura del file:

    \begin{tcolorbox}[breakable, size=fbox, boxrule=1pt, pad at break*=1mm,colback=cellbackground, colframe=cellborder]
\prompt{In}{incolor}{ }{\boxspacing}
\begin{Verbatim}[commandchars=\\\{\}]
\PY{n}{pdfFileObj} \PY{o}{=} \PY{n+nb}{open}\PY{p}{(}\PY{l+s+sa}{r}\PY{l+s+s1}{\PYZsq{}}\PY{l+s+s1}{/content/drive/MyDrive/ontologiaPDF.pdf}\PY{l+s+s1}{\PYZsq{}}\PY{p}{,} \PY{n}{mode}\PY{o}{=}\PY{l+s+s1}{\PYZsq{}}\PY{l+s+s1}{rb}\PY{l+s+s1}{\PYZsq{}}\PY{p}{)}
\PY{n}{PdfLeggi} \PY{o}{=} \PY{n}{PdfFileReader}\PY{p}{(}\PY{n}{pdfFileObj}\PY{p}{,} \PY{n}{strict}\PY{o}{=}\PY{k+kc}{False}\PY{p}{)}
\end{Verbatim}
\end{tcolorbox}

    Convertiamo le singole pagine del pdf in stringhe python che sono poi
unite nella stringa testo:

    \begin{tcolorbox}[breakable, size=fbox, boxrule=1pt, pad at break*=1mm,colback=cellbackground, colframe=cellborder]
\prompt{In}{incolor}{ }{\boxspacing}
\begin{Verbatim}[commandchars=\\\{\}]
\PY{n}{pages\PYZus{}number} \PY{o}{=} \PY{l+m+mi}{5}
\PY{n}{pagina} \PY{o}{=} \PY{n+nb}{list}\PY{p}{(}\PY{p}{)}
\PY{n}{testo\PYZus{}pagina} \PY{o}{=} \PY{n+nb}{list}\PY{p}{(}\PY{p}{)}
\PY{n}{testo} \PY{o}{=} \PY{l+s+s1}{\PYZsq{}}\PY{l+s+s1}{\PYZsq{}}
\PY{k}{for} \PY{n}{i} \PY{o+ow}{in} \PY{n+nb}{range}\PY{p}{(}\PY{n}{pages\PYZus{}number}\PY{p}{)}\PY{p}{:}
  \PY{n}{pagina}\PY{o}{.}\PY{n}{append}\PY{p}{(}\PY{n}{PdfLeggi}\PY{o}{.}\PY{n}{getPage}\PY{p}{(}\PY{n}{i}\PY{p}{)}\PY{p}{)}
  \PY{n}{testo\PYZus{}pagina}\PY{o}{.}\PY{n}{append}\PY{p}{(}\PY{n}{pagina}\PY{p}{[}\PY{n}{i}\PY{p}{]}\PY{o}{.}\PY{n}{extractText}\PY{p}{(}\PY{p}{)}\PY{p}{)}
  \PY{n}{testo} \PY{o}{=} \PY{n}{testo} \PY{o}{+} \PY{n}{testo\PYZus{}pagina}\PY{p}{[}\PY{n}{i}\PY{p}{]}
\end{Verbatim}
\end{tcolorbox}

    \hypertarget{estrazione-di-informazioni-da-testo}{%
\section{Estrazione di informazioni da
testo}\label{estrazione-di-informazioni-da-testo}}

    Ci occupiamo ora delle elaborazione del testo precedentemente caricato
utilizzando la libreria nltk, una libreria etremamente potente che ci
fornisce tutti gli strumenti fondamentali per il NLP

    \hypertarget{implementazione-di-una-pipeline-nlp-in-python}{%
\subsection{Implementazione di una Pipeline NLP in
Python}\label{implementazione-di-una-pipeline-nlp-in-python}}

    Come fase iniziale di pretrattamento del testo andiamo ad effettuare
delle operazioni di parsing: in particolarmodo eseguiamo un'operazione
di tokenizzazione scomponendo il testo in sigole parole o nelle singole
frasi che lo compongono.

    \begin{tcolorbox}[breakable, size=fbox, boxrule=1pt, pad at break*=1mm,colback=cellbackground, colframe=cellborder]
\prompt{In}{incolor}{ }{\boxspacing}
\begin{Verbatim}[commandchars=\\\{\}]
\PY{k+kn}{import} \PY{n+nn}{nltk}
\PY{n}{nltk}\PY{o}{.}\PY{n}{download}\PY{p}{(}\PY{l+s+s1}{\PYZsq{}}\PY{l+s+s1}{all}\PY{l+s+s1}{\PYZsq{}}\PY{p}{,} \PY{n}{quiet}\PY{o}{=}\PY{k+kc}{True}\PY{p}{)}
\PY{n}{token} \PY{o}{=} \PY{n}{nltk}\PY{o}{.}\PY{n}{word\PYZus{}tokenize}\PY{p}{(}\PY{n}{testo}\PY{p}{)}
\PY{n}{frasi} \PY{o}{=} \PY{n}{nltk}\PY{o}{.}\PY{n}{sent\PYZus{}tokenize}\PY{p}{(}\PY{n}{testo}\PY{p}{)}
\end{Verbatim}
\end{tcolorbox}

    Segue poi la fase di annotazione sintattica: sfruttando il comando
pos\_tag della libreria nltk andiamo ad effettuare il tagging
grammaticale, associando ad ogni token prodotto nella fase precedente la
sua categoria grammaticale:

    \begin{tcolorbox}[breakable, size=fbox, boxrule=1pt, pad at break*=1mm,colback=cellbackground, colframe=cellborder]
\prompt{In}{incolor}{ }{\boxspacing}
\begin{Verbatim}[commandchars=\\\{\}]
\PY{n}{POS\PYZus{}tag} \PY{o}{=} \PY{n}{nltk}\PY{o}{.}\PY{n}{pos\PYZus{}tag}\PY{p}{(}\PY{n}{token}\PY{p}{)}
\PY{n+nb}{print}\PY{p}{(}\PY{n}{POS\PYZus{}tag}\PY{p}{)}
\end{Verbatim}
\end{tcolorbox}

    Ci occupiamo a questo punto della stemmatizzazione, andando ad associare
ad ogni parola del testo il suo stem, utilizzando lo SNowballStemmer
fornito da nltk:

    \begin{tcolorbox}[breakable, size=fbox, boxrule=1pt, pad at break*=1mm,colback=cellbackground, colframe=cellborder]
\prompt{In}{incolor}{ }{\boxspacing}
\begin{Verbatim}[commandchars=\\\{\}]
\PY{k+kn}{from} \PY{n+nn}{nltk}\PY{n+nn}{.}\PY{n+nn}{stem} \PY{k+kn}{import} \PY{n}{SnowballStemmer} 
\PY{n}{radice} \PY{o}{=} \PY{n}{SnowballStemmer}\PY{p}{(}\PY{l+s+s1}{\PYZsq{}}\PY{l+s+s1}{english}\PY{l+s+s1}{\PYZsq{}}\PY{p}{)}
\PY{c+c1}{\PYZsh{} radice.stem(token[10]) }
\PY{k}{for} \PY{n}{word} \PY{o+ow}{in} \PY{n}{token}\PY{p}{:}
  \PY{n}{stem} \PY{o}{=} \PY{n}{radice}\PY{o}{.}\PY{n}{stem}\PY{p}{(}\PY{n}{word}\PY{p}{)}
  \PY{n+nb}{print}\PY{p}{(}\PY{l+s+s1}{\PYZsq{}}\PY{l+s+s1}{parola }\PY{l+s+si}{\PYZob{}\PYZcb{}}\PY{l+s+s1}{, stem: }\PY{l+s+si}{\PYZob{}\PYZcb{}}\PY{l+s+s1}{\PYZsq{}}\PY{o}{.}\PY{n}{format}\PY{p}{(}\PY{n}{word}\PY{p}{,} \PY{n}{stem}\PY{p}{)}\PY{p}{)}
\end{Verbatim}
\end{tcolorbox}

    Vogliamo procedere ora alla fase di lemmatizzazione: ad ogni parola del
testo vogliamo associare la sua forma di base. Per far ciò utilizzaimo
il WordNetLemmatizer che sfrutta un database in internet. Prima però
dobbiamo definire una funzione che mappi i PosTag estratti in precedenza
nel corretto formato previsto dal lemmatizzatore:

    \begin{tcolorbox}[breakable, size=fbox, boxrule=1pt, pad at break*=1mm,colback=cellbackground, colframe=cellborder]
\prompt{In}{incolor}{ }{\boxspacing}
\begin{Verbatim}[commandchars=\\\{\}]
\PY{k+kn}{from} \PY{n+nn}{nltk}\PY{n+nn}{.}\PY{n+nn}{corpus} \PY{k+kn}{import} \PY{n}{wordnet}
\PY{k+kn}{from} \PY{n+nn}{nltk}\PY{n+nn}{.}\PY{n+nn}{stem} \PY{k+kn}{import} \PY{o}{*}

\PY{n}{lemmatizzatore} \PY{o}{=} \PY{n}{WordNetLemmatizer}\PY{p}{(}\PY{p}{)}

\PY{k}{def} \PY{n+nf}{map\PYZus{}postag\PYZus{}to\PYZus{}lemtag}\PY{p}{(}\PY{n}{treebank\PYZus{}tag}\PY{p}{)}\PY{p}{:}

    \PY{k}{if} \PY{n}{treebank\PYZus{}tag}\PY{o}{.}\PY{n}{startswith}\PY{p}{(}\PY{l+s+s1}{\PYZsq{}}\PY{l+s+s1}{J}\PY{l+s+s1}{\PYZsq{}}\PY{p}{)}\PY{p}{:}
        \PY{k}{return} \PY{n}{wordnet}\PY{o}{.}\PY{n}{ADJ}
    \PY{k}{elif} \PY{n}{treebank\PYZus{}tag}\PY{o}{.}\PY{n}{startswith}\PY{p}{(}\PY{l+s+s1}{\PYZsq{}}\PY{l+s+s1}{V}\PY{l+s+s1}{\PYZsq{}}\PY{p}{)}\PY{p}{:}
        \PY{k}{return} \PY{n}{wordnet}\PY{o}{.}\PY{n}{VERB}
    \PY{k}{elif} \PY{n}{treebank\PYZus{}tag}\PY{o}{.}\PY{n}{startswith}\PY{p}{(}\PY{l+s+s1}{\PYZsq{}}\PY{l+s+s1}{N}\PY{l+s+s1}{\PYZsq{}}\PY{p}{)}\PY{p}{:}
        \PY{k}{return} \PY{n}{wordnet}\PY{o}{.}\PY{n}{NOUN}
    \PY{k}{elif} \PY{n}{treebank\PYZus{}tag}\PY{o}{.}\PY{n}{startswith}\PY{p}{(}\PY{l+s+s1}{\PYZsq{}}\PY{l+s+s1}{R}\PY{l+s+s1}{\PYZsq{}}\PY{p}{)}\PY{p}{:}
        \PY{k}{return} \PY{n}{wordnet}\PY{o}{.}\PY{n}{ADV}
    \PY{k}{elif} \PY{n}{treebank\PYZus{}tag}\PY{o}{.}\PY{n}{startswith}\PY{p}{(}\PY{l+s+s1}{\PYZsq{}}\PY{l+s+s1}{S}\PY{l+s+s1}{\PYZsq{}}\PY{p}{)}\PY{p}{:}
        \PY{k}{return} \PY{n}{wordnet}\PY{o}{.}\PY{n}{ADJ\PYZus{}SAT}
    \PY{k}{else}\PY{p}{:}
        \PY{k}{return} \PY{n}{wordnet}\PY{o}{.}\PY{n}{NOUN}
\end{Verbatim}
\end{tcolorbox}

    Eseguiamo ora un loop che, ad ogni parola, e corrispondente POS\_tag,
associ il corrispondente lemma:

    \begin{tcolorbox}[breakable, size=fbox, boxrule=1pt, pad at break*=1mm,colback=cellbackground, colframe=cellborder]
\prompt{In}{incolor}{ }{\boxspacing}
\begin{Verbatim}[commandchars=\\\{\}]
\PY{k}{for} \PY{n}{i}\PY{p}{,}\PY{n}{j} \PY{o+ow}{in} \PY{n}{POS\PYZus{}tag}\PY{p}{:}
  \PY{n}{pos} \PY{o}{=} \PY{n}{map\PYZus{}postag\PYZus{}to\PYZus{}lemtag}\PY{p}{(}\PY{n}{j}\PY{p}{)}
  \PY{n}{lemma} \PY{o}{=} \PY{n}{lemmatizzatore}\PY{o}{.}\PY{n}{lemmatize}\PY{p}{(}\PY{n}{i}\PY{o}{.}\PY{n}{lower}\PY{p}{(}\PY{p}{)}\PY{p}{,}\PY{n}{pos}\PY{o}{.}\PY{n}{lower}\PY{p}{(}\PY{p}{)}\PY{p}{)}
  \PY{n+nb}{print}\PY{p}{(}\PY{l+s+s1}{\PYZsq{}}\PY{l+s+s1}{parola: }\PY{l+s+si}{\PYZob{}\PYZcb{}}\PY{l+s+s1}{, lemma: }\PY{l+s+si}{\PYZob{}\PYZcb{}}\PY{l+s+s1}{\PYZsq{}}\PY{o}{.}\PY{n}{format}\PY{p}{(}\PY{n}{i}\PY{p}{,} \PY{n}{lemma}\PY{p}{)}\PY{p}{)}
\end{Verbatim}
\end{tcolorbox}

    Ci occupiamo a questo punto del chunking: cerchiamo all'interno del
testo diversi sintagmi nominali definiti da delle regole ``grammar'':

    Cerchiamo sintagmi composti da articolo, nome e aggettivo:

\begin{itemize}
\tightlist
\item
  le regole sono composte con il seguente formato:
  grammatica=``etichetta:\{\textless POS1\textgreater\textless POS2\textgreater\textless POS3\textgreater\}''
\end{itemize}

    \begin{tcolorbox}[breakable, size=fbox, boxrule=1pt, pad at break*=1mm,colback=cellbackground, colframe=cellborder]
\prompt{In}{incolor}{ }{\boxspacing}
\begin{Verbatim}[commandchars=\\\{\}]
\PY{n}{grammar} \PY{o}{=} \PY{l+s+s2}{\PYZdq{}}\PY{l+s+s2}{SintagmaNominale:}\PY{l+s+s2}{\PYZob{}}\PY{l+s+s2}{\PYZlt{}DT\PYZgt{}\PYZlt{}JJ\PYZgt{}\PYZlt{}NN\PYZgt{}\PYZcb{}}\PY{l+s+s2}{\PYZdq{}}
\PY{n}{cp} \PY{o}{=} \PY{n}{nltk}\PY{o}{.}\PY{n}{RegexpParser}\PY{p}{(}\PY{n}{grammar}\PY{p}{)} 
\PY{n}{sintagmiNominali} \PY{o}{=} \PY{n}{cp}\PY{o}{.}\PY{n}{parse}\PY{p}{(}\PY{n}{POS\PYZus{}tag}\PY{p}{)}
\PY{n+nb}{print}\PY{p}{(}\PY{n}{sintagmiNominali}\PY{p}{)}
\end{Verbatim}
\end{tcolorbox}

    Cerchiamo ora sintagmi più complessi definendo una regola grammaticale
condizionale:

\begin{itemize}
\tightlist
\item
  ? se inserito permette di considerare o meno la presenza del tipo di
  token ricercato
\item
  * può significare 0 volte a tante volte
\item
  + significa almeno una volta oppure tante volte
\end{itemize}

    \begin{tcolorbox}[breakable, size=fbox, boxrule=1pt, pad at break*=1mm,colback=cellbackground, colframe=cellborder]
\prompt{In}{incolor}{ }{\boxspacing}
\begin{Verbatim}[commandchars=\\\{\}]
\PY{n}{grammar2} \PY{o}{=} \PY{l+s+s2}{\PYZdq{}}\PY{l+s+s2}{SintagmaNominale2:}\PY{l+s+s2}{\PYZob{}}\PY{l+s+s2}{\PYZlt{}DT\PYZgt{}?\PYZlt{}JJ\PYZgt{}*\PYZlt{}NN\PYZgt{}+\PYZcb{}}\PY{l+s+s2}{\PYZdq{}}
\PY{n}{cp2} \PY{o}{=} \PY{n}{nltk}\PY{o}{.}\PY{n}{RegexpParser}\PY{p}{(}\PY{n}{grammar2}\PY{p}{)}
\PY{n}{sintagmiNominali2} \PY{o}{=} \PY{n}{cp2}\PY{o}{.}\PY{n}{parse}\PY{p}{(}\PY{n}{POS\PYZus{}tag}\PY{p}{)}
\PY{n+nb}{print}\PY{p}{(}\PY{n}{sintagmiNominali2}\PY{p}{)}
\end{Verbatim}
\end{tcolorbox}

    Cerchiamo sintagmi composti da nome + verbo:

    \begin{tcolorbox}[breakable, size=fbox, boxrule=1pt, pad at break*=1mm,colback=cellbackground, colframe=cellborder]
\prompt{In}{incolor}{ }{\boxspacing}
\begin{Verbatim}[commandchars=\\\{\}]
\PY{n}{grammar3} \PY{o}{=} \PY{l+s+s2}{\PYZdq{}}\PY{l+s+s2}{SintagmaNominale3:}\PY{l+s+s2}{\PYZob{}}\PY{l+s+s2}{\PYZlt{}NN\PYZgt{}\PYZlt{}VB.*\PYZgt{}\PYZcb{}}\PY{l+s+s2}{\PYZdq{}}
\PY{n}{cp3} \PY{o}{=} \PY{n}{nltk}\PY{o}{.}\PY{n}{RegexpParser}\PY{p}{(}\PY{n}{grammar3}\PY{p}{)}
\PY{n}{sintagmiNominali3} \PY{o}{=} \PY{n}{cp3}\PY{o}{.}\PY{n}{parse}\PY{p}{(}\PY{n}{POS\PYZus{}tag}\PY{p}{)}
\PY{n+nb}{print}\PY{p}{(}\PY{n}{sintagmiNominali3}\PY{p}{)}
\end{Verbatim}
\end{tcolorbox}

    Cerchiamo in fine sintagmi di diversa natura con una regola or:

    \begin{tcolorbox}[breakable, size=fbox, boxrule=1pt, pad at break*=1mm,colback=cellbackground, colframe=cellborder]
\prompt{In}{incolor}{ }{\boxspacing}
\begin{Verbatim}[commandchars=\\\{\}]
\PY{n}{grammar4} \PY{o}{=} \PY{l+s+sa}{r}\PY{l+s+s2}{\PYZdq{}\PYZdq{}\PYZdq{}}
\PY{l+s+s2}{Sintagma3elementiArtAggNom:}\PY{l+s+s2}{\PYZob{}}\PY{l+s+s2}{\PYZlt{}DT\PYZgt{}?\PYZlt{}JJ\PYZgt{}*\PYZlt{}NN\PYZgt{}+\PYZcb{}}
\PY{l+s+s2}{Sintagma2elementiNomeVerbo:}\PY{l+s+s2}{\PYZob{}}\PY{l+s+s2}{\PYZlt{}NN\PYZgt{}\PYZlt{}VB\PYZgt{}\PYZcb{} }
\PY{l+s+s2}{Sintagma1elementoVerbo:}\PY{l+s+s2}{\PYZob{}}\PY{l+s+s2}{\PYZlt{}VB.*\PYZgt{}*\PYZcb{}}
\PY{l+s+s2}{\PYZdq{}\PYZdq{}\PYZdq{}}

\PY{n}{cp4} \PY{o}{=} \PY{n}{nltk}\PY{o}{.}\PY{n}{RegexpParser}\PY{p}{(}\PY{n}{grammar4}\PY{p}{)}
\PY{n}{sintagmiNominali4} \PY{o}{=} \PY{n}{cp4}\PY{o}{.}\PY{n}{parse}\PY{p}{(}\PY{n}{POS\PYZus{}tag}\PY{p}{)}
\PY{n+nb}{print}\PY{p}{(}\PY{n}{sintagmiNominali4}\PY{p}{)}
\end{Verbatim}
\end{tcolorbox}

    \hypertarget{ricerca-sinomini-termini-correlati}{%
\section{Ricerca sinomini, termini
correlati}\label{ricerca-sinomini-termini-correlati}}

    Cerchiamo dal database wordnet i sinonimi di una parola che sostituiti
nel testo non ne stravolgono il significato:

    \begin{tcolorbox}[breakable, size=fbox, boxrule=1pt, pad at break*=1mm,colback=cellbackground, colframe=cellborder]
\prompt{In}{incolor}{ }{\boxspacing}
\begin{Verbatim}[commandchars=\\\{\}]
\PY{k+kn}{from} \PY{n+nn}{nltk}\PY{n+nn}{.}\PY{n+nn}{corpus} \PY{k+kn}{import} \PY{n}{wordnet} \PY{k}{as} \PY{n}{wn}
\PY{n}{output} \PY{o}{=} \PY{n}{wn}\PY{o}{.}\PY{n}{synsets}\PY{p}{(}\PY{l+s+s1}{\PYZsq{}}\PY{l+s+s1}{home}\PY{l+s+s1}{\PYZsq{}}\PY{p}{,}\PY{l+s+s1}{\PYZsq{}}\PY{l+s+s1}{a}\PY{l+s+s1}{\PYZsq{}}\PY{p}{)}
\PY{n+nb}{print}\PY{p}{(}\PY{n}{output}\PY{p}{)}
\end{Verbatim}
\end{tcolorbox}

    Cerchiamo in fine la descrizione ed una frase di esempio contenete la
parola di interesse:

    \begin{tcolorbox}[breakable, size=fbox, boxrule=1pt, pad at break*=1mm,colback=cellbackground, colframe=cellborder]
\prompt{In}{incolor}{ }{\boxspacing}
\begin{Verbatim}[commandchars=\\\{\}]
\PY{n}{parola\PYZus{}che\PYZus{}mi\PYZus{}interessa} \PY{o}{=} \PY{n}{wn}\PY{o}{.}\PY{n}{synset}\PY{p}{(}\PY{l+s+s1}{\PYZsq{}}\PY{l+s+s1}{home.s.03}\PY{l+s+s1}{\PYZsq{}}\PY{p}{)} 
\PY{n+nb}{print}\PY{p}{(}\PY{n}{parola\PYZus{}che\PYZus{}mi\PYZus{}interessa}\PY{o}{.}\PY{n}{definition}\PY{p}{(}\PY{p}{)}\PY{p}{)} 
\PY{n+nb}{print}\PY{p}{(}\PY{n}{parola\PYZus{}che\PYZus{}mi\PYZus{}interessa}\PY{o}{.}\PY{n}{examples}\PY{p}{(}\PY{p}{)}\PY{p}{)}
\end{Verbatim}
\end{tcolorbox}


    % Add a bibliography block to the postdoc
    
    
    
\end{document}
